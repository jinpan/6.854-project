\section{Heuristics}
In this section we present several heuristics that don't speed up the
theoretical running time but reduce the number of phases or shortest
path computations in practice.
\subsection{Demand scaling}
This heuristic allows us to assume that $\beta\geq 1$. Recall that
$\beta$ is defined to be the minimum value of
$\frac{D(i)}{\alpha(i)}$, and is necessary is achieving a bound on
$\gamma$, the ratio between the primal and dual solutions. That is,
the number of phases depends on $\beta$. Notice that scaling all
demands by a factor of $c$ scales $\alpha$ by a factor of $c^{-1}$, so
scaling all demands by $c$ scales $\beta$ by $c$. If we let $z_j$ be
the maximum possible flow of commodity $j$, we define $z$ to be the
maximum fraction of the demands that can be routed independently to
each commodity. Thus, $\beta$ is bounded above by $z$ and bounded
below by $\frac{z}{k}$. Thus, we scale each demand by $\frac{k}{z}$
such that $\frac{z'}{k} =1$ where $z'$ represents a calculation of $z$
with the updated demands. In doing this, we bound $\beta$ below by
1. We can then rescale demands accordingly if our procedure doesn't
stop within $\frac{2}{\epsilon}\log_{1+\epsilon}\frac{m}{1-\epsilon}$
based on equation EQ [update this later], which implies that
$\beta\geq 2$, so scaling demands by a factor of 2 maintains
$\beta\geq 1$. This multiplies the total number of phases by a factor
o $\log k$ and costs $k\cdot T_{mf}$ where $T_{mf}$ is the time to
compute a max flow.
\subsection{2-approximation for Beta estimation}
This heuristic allows us to greatly reduce the number of phases we
perform by bounding $\beta$ between 1 and 2. To do this, we first
compute a 2-approximation to the problem using $O(\log k \log m)$
phases, and returns a $\hat{\beta}$ such that $\beta\leq \hat{\beta}\leq2\beta$. Then we don't have to scale demands using the above procedure
and reduces the number of phases to $O(\log m(\log k+\epsilon^{-1})$.

\subsection{Karakosta's method for shared sources}
This heuristic allows us to remove the depedence upon $k$, the number
of commodities, using a technique first introduced by Karakosta [CITE]. Since
we compute shortest paths using a single-source shortest path method,
with no additional runtime we obtain the minimum cost path from the
source to all possible sinks. Grouping the commodities by shared
sources, we can run the iterations for every commodity in a particular
group at the same time. To do this we make only one call to Djikstra's
for each group for each step and route a scaled flow along the path
associated with each commodity. We scale each flow for commodity $j$
by the ratio of the demand remaining for commodity $j$ to the sum of
demands remaining for all other commodities within this group. The
above analysis still holds since we scale at least one edge by a
factor of $1+\epsilon$ for each step except the last. The number of
iterations then, is bounded by the number of groups, which is at most
$n$. Thus our runtime of $O(\omega^{-2}(k\log m+m)\log m T_{sp})$
becomes $O(\omega^{-2}(m\log m+m)\log m T_{sp})$, or $\~{O}(w^{-2}m^2)$.

\subsection{Multiple routing}
We'll fill this in once we implement it.
