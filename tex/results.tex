\section{Experimental Results}

We have implemented Garg-MCF in the Python programming language and tested it upon a variety of input problems using the
PyPy JIT compiler. Due to the automatic garbage collection and high level nature of Python, the clock time of our
implementation was occasionally inconsistent and should be taken with a grain of salt. We report the number of calls to
the single-source shortest path subroutine as a more reliable performance indicator, as we make one call per phase of
the algorithm. We first describe the construction of random graphs.

We construct random directed graphs using the networkx module's $\mathrm{gnm\_random\_graph}$ routine, feeding
parameters of $n$ and $m$ \cite{networkx}. Then we randomly create commodities by a grouping procedure. We take the
number of commodities as a parameter and a distribution which defines the number of shared source commodities we have,
randomly choosing sinks for the commodities, ensuring that there exists a directed path from the source to the sink for
each commodity. Capacities and demands are chosen at random. This set of procedures allows us to randomly generate a
directed graph with a given number of nodes, edges, commodities, and we can explicitly specify the number of commodity
groups. With this in hand, we can proceed to test the algorithm's dependence on parameters \{$\omega$, $k$, $n$\},
taking note of the effect of Karakostas' heuristic for shared source commodities.

We first validated the correctness of our algorithms by running on
several small graphs $(m<\leq 10)$ with multiple commodities that
possessed maximum concurrent flow values that were computable by
hand. Included in these test cases were multiple commodities with
shared and unshared sources, and demands that vary orders of
magnitudes to empirically validate relaxation on demand-scaling. We
found our implementation provided an answer within the $(1+\omega)$
approximation ratio on all test cases and thus proceed confidently
with the following experiments.

\subsection{Dependence on the parameter $\omega$}

In theory, the Garg-MCE algorithm runs with a quadratic dependence on $\omega$, which we observe in practice, both in
terms of number of calls to our shortest paths function, and also wall time.

This data was obtained from running our algorithm, with the respective heuristics activated, on graphs with $100$
nodes, $400$ directed edges, 10 commodities (split into two groups of
size 6 and 4 that shared the same source) on 10 different random
graphs. Of the 10 measured shortest path computations for each
$\omega$, we dropped the minimum and maximum, and reported the average over the remaining.

We observe that the algorithm works correctly and even slightly more efficiently if we do \emph{not} scale $\beta$ such
that it is bounded above by 1, as previous analysis assumes.

The two-approximation heuristic was implemented on top of Karakostas' heuristic and we do not observe it outperforming
just Karakostas' heuristics. However, our data hints that the two-approximation may be asymptotically more optimal,
but, regrettably, limited precision arithmetic on primitive floats and expensive computations on arbitrary precision
decimals render testing smaller values of $\omega$ infeasible.

We include data on physical time performance in the appendix.

\subsection{Dependence on the parameter $k$}

The Garg-MCE algorithm has a linear dependence on k, and the Karakostas heuristic absorbs this factor.

% INSERT FIGURE
This data was obtained from running our algorithm, with the respective heuristics activated, on graphs with $100$
nodes, $400$ directed edges, and a error margin of $10\%$, 10 times.

Again, we observe our vanilla implementation outperforming the implementation with $\beta$ scaling, the Karakostas
heuristic outperforming both, and the two-approximation having similar performance.

\subsection{Dependence on the parameter $n$}

Our algorithm has a quadratic dependence on the number of edges, comparable to the number of nodes, given our sparse
graph.  We expect there to be a linear dependence of the number of shortest path calls on the number of nodes.

This data was obtained from running our algorithm, with the respective heuristics activated, 10 commodities (split into
two groups of size 6 and 4 that shared the same source), and an error margin of $10\%$, 10
times. We construct our graphs such that the number of edges is always four times the number of nodes.

A surprising result is that this dependence is experimentally not very strong for relatively small $n$: we observe that
the number of calls to shortest paths does not necessarily increase with an increase in graph size.

\subsection{Dependence on the distribution}

Our algorithm does not strictly depend on how the commodities are distributed, but the Karakostas heuristic takes
advantage of multiple commodities that start at the same source. We expect the Karakostas heuristic to significantly
outperform baseline in the case where all the commodities start at the same source, and underperform when all the
commodities start at differing sources.

This data was obtained from running our algorithm, with the respective heuristics activated, on graphs with $100$
nodes, $400$ directed edges, 10 commodities, 10
times. Each time, we vary the distribution of commodities that share the same source.

Somewhat surprisingly, our baseline vanilla and beta algorithms are dependent on the input distribution.  We hypothesize
that this may be due to underlying caching optimizations that work more efficiently when the search algorithm is more
predictable and visits the same nodes initially.


