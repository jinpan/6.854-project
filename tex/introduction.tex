\section{Introduction}

The multicommodity flow problem requires routing multiple commodities
from their respective sources to their respective sinks along a
directed graph. It also requires that the net flow across all commodities on any
single edge does not exceed the capacity of that edge and that flow is conserved. Multicommodity
flow problems arise in many different contexts where distinct
resources need to be routed across a network. For example,
multicommodity flow problems are solved when routing across
communication networks or determining transportation of goods.

Historically, this problem and its many variants have been expressed as a
large linear program. This allows us to use the multitude of linear programming 
algorithms to solve this problem exactly in polynomial time. The
structure of this problem has been used to modify
interior point methods to generate faster runtimes in practice, but
the size of the linear program quickly gets prohibitively large. In
practice, fully polynomial approximation schemes can very closely
approximate a solution to the multicommodity flow problem, even for
large problem instances. In practice, getting within $1\%$ or even
$5\%$ of the optimal solution is often good enough, and can be
attained much more quickly than with linear
programming methods. These fully polynomial approximation schemes
often rely on subroutines which can be efficiently computed in
practice, such as minimum cost flow or single-source shortest paths.

In this paper, we implement an algorithm for solving multicommodity
flow variants first introduced by Garg in 2007 [CITE].%\cite{garg}
Specifically, we offer
an implementation solving the maximum concurrent flow problem. We also
implement several heuristics that take advantage of certain features
of a problem instance to further speed up the algorithm in
practice. The paper is organized as follows. In section 2
we formally present the problem and define maximum concurrent flow in
addition to discussing previous contributions towards creating an
efficient fully polynomial approximation scheme for this problem. In
section 3, we present the algorithm and offer a brief analysis of its
runtime and correctness. In section 4, we present several heuristics
implemented to speed up runtime in practice. Section 5 contains our
description of our implementation, experiments, results and
discussion.  Finally we offer our concluding remarks in section 6.

