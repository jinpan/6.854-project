\section{Algorithm}
\subsection{Presentation of algorithm}
Avoiding the complicated methods of the most recent publication
improving the bounds for maximum concurrent flow, we choose to
implement the maximum concurrent flow FPAS presented by Garg and
K\"{o}nemann, and several heuristics to speed up runtime in
practice. The basic implementation gives a runtime of
$\~{O}(\omega^{-2}(k+m)m)$ for the MCF problem. The dependency on $k$
can be removed with
the implementation of our heuristics as described later. For the
remainder of this section, we will describe the algorithm implemented
in this paper (herein referred to as Garg-MCF), offering both
pseudocode and a brief look at the analysis. \\
The main idea behind Garg-MCF relies on the rerouting and fractional
packing methods introduced in section 2. The intuition behind Garg-MCF
comes from fractional packing techniques first introduced by Shahrokhi
and Matula. 

In plain english, we start by assigning a 'length' to each edge
dependent upon the error term within our approximation $\omega$ and
the capacity of that edge. Then we repeatedly satisfy the
demand of each commodity by sending as much flow as we can
(independently) along the
shortest path from the source to the sink, where shortest is dependent
upon the 'length' function. Then we scale the 'length' of each edge by
a factor dependent upon the amount of flow we just sent across it. In
this way, the 'length' is exponential in the amount of flow being sent
across the edge. Intuitively, since we call shortest paths based on
this length function, this causes us to spread our flow across edges
and reroute in such a fashion that doesn't force all our flow on one
path. We stop after the length functions grow 'large enough'. The
precise definition of 'large enough' and that this computes a $1+\omega$ approximation is not
entirely obvious, but is believable under the context of the
analysis. We will be more formal in how Garg-MCF works
now.

We refer to the second linear program formulation of maximum
concurrent flow, which we will refer to as P-MCF. Taking the dual of
P-MCF generates the following linear program, which we'll refer to as D-MCF, which defines a length
$l(e)$ for each edge and a variable $z(j)$ for each commodity.
\begin{align*}
\text{min     } \sum_{e\in E}c(e)l(e) \\
\text{s.t. }\sum_{e\in p}l(e) \geq z(j) \;\;\forall 1\leq j \leq k,
\forall p \in \cal{P}_j\\
\sum_{j=1}^kd(j)\cdot z(j)\geq 1\\
l,z\geq 0
\end{align*}
Recalling that $\mathcal{P}_j$ is the set of paths from $s_j$ to
$t_j$. Intuitively, the first constraint maintains that, when tight, $z(j)$ is the
value of the length of the shortest path from $s_j$ to $t_j$. Keeping
with Garg's notation, we define the objective value to be a function
of the length assignment $l$:
$$D(l) := \sum_{e\in E} c(e)l(e)$$
and define $\alpha(l)$ as the second constraint:
$$\alpha(l) := \sum_{j=1}^k d(j)\cdot z(j)$$

Then since we'll have a minimal objective value when the second
constraint is tight, this LP can be viewed as the assignment of
positive lengths to edges such that $\frac{D(l)}{\alpha(l)}$ is
minimized. 

With notation in hand, the algorithm runs as follows. We first assign
an initial length of $\delta/c(e)$ to each edge, where $\delta$ is carefully picked
to make the analysis work out. Next we proceed in phases. For each
phase, we loop through each commodity $j$ in a series of $k$
iterations, one for each commodity. For each $j^{th}$ iteration then, we
consider commodity $j$ and reroute $d(j)$ units of flow from $s_j$ to
$t_j$. We do this with a series of steps. Let $l_{i,j}^s$ refer to the
length function at the $i^{th}$ phase, the $j^{th}$ iteration,
directly after the $s^{th}$ step. Then during each step, we compute the minimum cost
path from $s_j$ to $t_j$ under this length function,$l_{i,j}^s$ and route as much
flow along that path as possible. That is, we wish to route a total of
$d(j)$ flow from $s_j$ to $t_j$ over all steps during an iteration, so
let $d^s$ refer to flow remaining to be rerouted during an
iteration, after $s$ steps. Initially $d^0=d(j)$ and conclude the
iteration when $d^q=0$ for some $q$. 

Then for each step, we compute a path $p$,
and we route $f_{i,j}^{s+1}$ flow along this path, where
$f_{i,j}^{s+1}$ is the maximum allowable flow we can send, or in other
words, the minimum between the capacity of the minimum capacity edge
in $p$ and the flow remaining to be sent during this iteration. So
$$f_{i,j}^{s+1}=\min(\min_{e\in p}(c(e)),d_{i,j}^s)$$
We then decrease the amount of flow remaining to be routed, $d^s$ by
$f_{i,j}^{s+1}$. Since we routed flow along every edge in $p$, we also
need to update the length function for these edges, so we do this by
multiplying $l_{i,j}^s(e)$ by $1+\epsilon \frac{f_{i,j}^{s+1}}{c(e)}$ for an
$\epsilon$ that we define later, dependent upon $\omega$. 

We terminate
the iteration when $d_{i,j}^p=0$ for some step number $p$. We repeat
this process for each commodity during an iteration, and we repeat
phases until we reach our stopping condition, which we define when 
$D(l_{i,j}^s)\geq 1$. We then have a graph that has flow along edges, but
is almost surely infeasible. We can scale the flow along each edge by
dividing flow by $\log_{1+\epsilon}\frac{1}{\delta}$, which makes the
flow feasible. From here, we can calculate $\lambda$
directly. Motivation for calculation of $\delta$, $\epsilon$, and the
final scale factor, $\log_{1+\epsilon}\frac{1}{\delta}$, will be
briefly described in the next section, and fully derived in the
original paper. 
For clarity, we present the pseudocode of this
algorithm:

\SetAlgoSkip{}
%%%%%%%%%%%%%%%%%%%%%%%%%%%%%%%%%%%%%%%%%%%%%%%%%%%%%%%%%%%%%%%%%%%%%%%%%%%%%%%%%%%%%%%%%%
\begin{algorithm}[H] \small
$G \leftarrow (V,E)$\;
$\epsilon,\delta \leftarrow \mathrm{calculate\_epsilon}(\omega),\mathrm{calculate\_delta}(\omega)$\;
\For{ $e \in E$}{
  $l(e)=\delta / c(e)$\;
}
$\mathrm{scale\_demands}(\frac{k}{z})$\;
$phase\_since\_rescale \leftarrow 0$\;
\While{$D(l)<1$}{
  \If{ $phase\_since\_rescale>T$}{
  $\mathrm{scale\_demands}(2)$\;
  $phase\_since\_rescale \leftarrow 0$\;
}
  \For{ $j = 1$ to $k$}{
    $d_j \leftarrow d(j)$\;
    \While{$d_j \neq 0$} {
      $p \leftarrow \mathrm{shortest\_path(s_j,t_j)}$\;
      $min\_cap \leftarrow \infty$\;
      \For{$e\in p$}{
        $min\_cap \leftarrow \min(c(e),min\_cap)$\;
      }
      $f \leftarrow \min(min\_capacity,d_j)$\;
      $d_j \leftarrow d_j - f$\;
      \For{$e \in p$}{
        $e.flow \leftarrow e.flow+f$\;
        $l(e) \leftarrow l(e) \cdot (1+\epsilon \cdot f / c(e))$\;
        }

      }
    }
    $phase\_since\_rescale +=1$\;
}
$scale\_factor \leftarrow \log_{1+\epsilon}\delta^{-1}$\;
\For{$e\in E$}{
  $e.flow = e.flow / scale\_factor$\;
  }

\Return{$\mathrm{calculate\_lambda(G)}$}\;
\caption{GARG-MCF without heuristics}
\end{algorithm}
%
% Restore default vertical spacing
\SetAlgoSkip{smallskip}


\section{Analysis}
We offer a brief, high-level overview of the analysis. We point the
reader to the original paper for a more in-depth analysis.
\subsection{Approximation Ratio}
To prove the correctness of Garg-MCF, we care about approximating a
feasible solution to the dual LP, D-MCF. The key idea here is compare
our calculated $\lambda$ value to the best-possible solution to the
dual LP, D-MCF. Bounding this ratio above by $(1+\omega)$ will attain
the appropriate approximation ratio, since the ratio is bounded below
by 1, acconding to weak duality. We'll briefly describe how this is
done below.

We first assume that the
optimal objective value, $\beta$, will be at least 1. We can remove this
assumption later. Then we notice a relation between the objective
value at phase $i$ and phase $i-1$ and use this to establish a bound
on the optimal objective value divided by the number of phases we must
run through to reach our stopping condition. We next can create a
lower bound for $\lambda$ using the knowledge of how much flow we must
route through each edge up to the penultimate phase. It is here that
our scaling factor is derived, and a lower bound for $\lambda$
established dependent on the number of phases completed and
$\epsilon$. 

Now we can consider the ratio between $\lambda$ and the solution to
D-MCF, since we have bounds on both terms. If we can show that
$\beta/\lambda\leq \text{poly}(\epsilon)$ then by weak duality we have
that $\lambda \leq \beta$, so our computed $\lambda$ is within a
factor of $\text{poly}(\epsilon)$ of $\beta$. Strong duality holds
that $\beta$ is the value of the optimal solution to MCF. With the
math of the paper  we arrive at the claim that 
$$\frac{\beta}{\lambda}\leq (1-\epsilon)^{-3}$$
Thus, if we choose $\epsilon$ such that 
$$(1-\epsilon)^{-3}\leq 1+\omega$$ 
we arrive at our desired $1+\omega$ approximation.
\subsection{Scaling Beta}
We now lift the previous assumption that $\beta\geq 1$. The key idea
here is that by scaling all the demands by a constant factor, we also
scale the value of the optimal $\lambda$, and therefore $\beta$. We
find bounds above and below for $\beta$ based on the current demand
scheme, and then scale the demands such that the lower bound for
$\beta$ is 1. It turns out that for any phase number $i$ in which the
algorithm has not yet terminated, $i$ is strictly less than
$\frac{\beta}{\epsilon}\log_{1+\epsilon}\frac{m}{1-\epsilon}$. Then we
can run phases until we have computed enough
phases to ensure that $\beta\geq 2$, a number we'll call $T$. In this case
we scale demands of
all commodities, effectively reducing $\beta$ by a factor of
two. If the ratio of the upper bound and lower bound for $\beta$ was
initially $c$, then we have to run for at most $T\log c$ phases.
\subsection{Running time}
The scaling of $\beta$ gives us a bound on the number of phases we
must compute. In each phase, we compute $k$ iterations, so all that is
left is to bound the number of steps per iteration. The key idea here
is that $l(e)c(e)$ is $\delta$ for each edge $e$ initially, and no
more than $1+\epsilon$ at termination, since we terminate as soon as
$\sum_el(e)c(e)\geq 1$ and we scale each edge by no more than
$1+\epsilon$ each time we scale. Thus the number of steps is at most
$m\log_{1+\epsilon}\frac{m}{1-\epsilon}$ plus the number of
iterations. Putting this all together, we end up with a runtime of 
$$O(\omega^{-2}(k\log m+m)\log m)T_{sp})$$ where $T_{sp}$ is the time to
compute a single-source shortest path subroutine.



