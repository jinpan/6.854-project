\section{Background}
Given a directed graph $G(V,E)$ with edge capacities $c: e \rightarrow
\mathbb{R}^+$, and k commodities with source $s_j$ and sink $t_j$ for
commodity $j$. Each commodity also has an associated demand
$d(j)$. The problem of maximum concurrent flow is to find a feasible flow that
maximizes $\lambda$, where for a given flow $f$ that routes $f_j$
units of commodity $j$ from $s_j$ to $t_j$, 
$$\lambda = \min_{j}(f_j/d(j)$$
We remember that a feasible flow preserves capacity constraints: 
$$\sum_j f_j(u,v) \leq c(u,v)$$
where $f_j(u,v)$ represents the flow of commodity $j$ through edge
$(u,v)$ for each edge $((u,v))\in E$, and maintains flow conservation:
$$\sum_{v\in V} f_j(u,v)=0$$ 
for $v\neq s_j,t_j$, for all commodities $j$ and defining
$f_j(u,v)=-f_j(v,u)$. Intuitively, we can describe the maximum
concurrent flow problem as the following situation: we want to send
commodities to their respective sinks, but instead of shooting for the
bare minimum of demand satisfaction, we want to maximize the ratio of
supply to demand for all commodities. \\
Typically this problem is formulated under the context of a linear
program. If we let $\cal{F}_j$ be the set of flows that send d(j) units
of commodity j from $s_j$ to $t_j$, letting $\cal{F}$ be the union of
$\cal{F}_j$ for $1\leq j \leq k$, we can formulate this as the following LP:
\begin{align*}
\text{max     } \lambda \\
\text{s.t. }\sum_{f\in \cal{F}}f_e\cdot x(f) \leq c(e) \;\;\forall
e\in E \\
\sum_{f\in \cal{F}_j} x(f)\geq \lambda \;\;\;\forall 1\leq j\leq k \\
x\geq 0,\; \lambda\geq 0.
\end{align*}
In this case, $x(f)$ is defined as the amount of times we use the
primitive flow $f$, sending $d(j)$ units of commodity $j$ from $s_j$
to $t_j$ for some $j$. Alternatively, we can model this problem with a
different LP formulation, operating on paths instead of flows. Let
$\cal{P}_e$ be the set of paths starting at $s_j$ and ending at $t_J$
for some $j$ that contain edge $e$, and let
$\cal{P}_j$ be the set of paths from $s_j$ to $t_j$. Our LP
formulation is as follows:
\begin{align*}
\text{max     } \lambda \\
\text{s.t. }\sum_{p\in \cal{P}_e}x(p) \leq c(e) \;\;\forall
e\in E \\
\sum_{p\in \cal{P}_j} x(p)\geq \lambda\cdot d(j) \;\;\;\forall 1\leq j\leq k \\
x\geq 0,\; \lambda\geq 0.
\end{align*}
In this case, $x(p)$ can be defined as the amount of flow we send
along path $p$. Both formulations solve an equivalent problem, however
it is clear that both formulations are exponential in size. For this
reason many fully polynomial approximation schemes have been
developed. We will now briefly discuss previous algorithms and recent
work on this problem.\\
The first fully polynomial approximation schemes (FPAS) for solving
multi-commodity flow problems and their variants arose in the early
90's with Leighton et. al [CITE]. Since the story of min-cost
multi-commodity flow (MCMCF) and maximum concurrent flow (MCF) are so intertwined,
we will describe the background of MCMCF, keeping in mind that FPAS'
for MCMCF are easily extensible to MCF, often using the same tricks,
both in theory and to speed up implmentation in
practice.\\ Theoretically, these algorithms run faster than
interior-point methods for solving LP's, but it was several years
after the development of the theoretical development of
the first FPAS's that an efficient implementation was developed,
attaining a speed two-to-three orders of magnitude faster than
state-of-the-art linear program solvers [CITE]. The main idea of how
these early FPAS' work is a rerouting method, generalizing on
fractional packing techniques[CITE-karger abstract]. Initially, the
algorithm finds an initial flow satisfying the demands, but possibly
violating the capacities. Then the algorithm repeatedly
picks a commodity via round-robin fashion and computes a
single-commodity minimum-cost flow in
the auxiliary graph, where the arcs have 'cost' that is exponential in
the current flow through the graph. A fraction of this commodity's
flow is rerouted to the newly computed minimum-cost flow. The
'goodness' of the reroutings are stored in a potential function, which
is guaranteed to generate a $1+\omega$ solution in
$\~{O}(\omega^{-3}kmn)$ time for the minimum cost multi-commodity flow
problem [CITE]. This was soon reduced to a quadratic dependence on
$\omega$. 
Since then, these bounds have since been increased to
$\~{O}(\omega^{-2}m^2)$ in 2000[Cite]. The current state-of-the-art
bound for maximum concurrent multicommodity flow in $O(k^2 \omega^2
2^{O(\sqrt(\log|V| \log\log |V|))})$, using a combination of a
non-Euclidean generalization of gradient descent, flow sparsifiers,
and an $O(m^{o(1)})$-competitive oblivious routing scheme [cite].
